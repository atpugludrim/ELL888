\documentclass[11pt]{article}

%FOR SCRIBES: Please change the next three lines to reflect the correct
%FOR SCRIBES: lecture number, name, and date.
\newcommand{\lecturenumber}{0}
\newcommand{\scribename}{{Mridul Gupta}}
\newcommand{\lecturedate}{29 March 2022} 

\usepackage{subfigure}
\usepackage{color}
\usepackage{url}
\usepackage{graphicx}
\usepackage{fullpage}
\usepackage{mathtools}
\newcommand{\etal}{{\em et al.}}
\newcommand{\qed}{\mbox{}\hspace*{\fill}\nolinebreak\mbox{$\rule{0.6em}{0.6em}$}}
\newcommand{\expect}{{\bf \mbox{\bf E}}}
\newcommand{\prob}{{\bf \mbox{\bf Pr}}}

%--------------------------- Commands and Environments I added -----------------
\usepackage[english]{babel}
\usepackage{amssymb}
\usepackage{amsmath}
\usepackage{fancyhdr}
\renewcommand{\baselinestretch}{1.10}
%%      Fonts:
%%---------------------------------------------------------------------------
\newfont{\bssten}{cmssbx10}
\newfont{\bssnine}{cmssbx10 scaled 900}
\newfont{\bssdoz}{cmssbx10 scaled 1200}

%---------------------------------------------------------------------------
\newcounter{topic} \setcounter{topic}{0}
\newcommand{\topic}[1]{\par \refstepcounter{topic} {\bssdoz \arabic{topic}.~ #1} \par}
%\newcommand{\topic}[1]{\par \refstepcounter{topic} \vs{2ex} {\bssdoz \arabic{topic}.~ #1} \par \vs{1ex}}

%------------------------------ end of new commands and evironments ------------

\definecolor{gray}{rgb}{0.5,0.5,0.5}
\newcommand{\comment}[1]{{\color{gray}[\textsf{#1}]}}
\newcommand{\redospace}{\small\renewcommand{\baselinestretch}{1.5}\normalsize}
\newcommand{\undospace}{\small\renewcommand{\baselinestretch}{1}\normalsize}
\newtheorem{theorem}{Theorem}[section]
\newtheorem{lemma}[theorem]{Lemma}
%----------------------------- some other things I added ---------------------
\newtheorem{claim}[theorem]{Claim}
\newtheorem{example}[theorem]{Example}
\newtheorem{protocol}[theorem]{Protocol}
%----------------------------------------------------------------------------
\newtheorem{corollary}[theorem]{Corollary}
\newtheorem{definition}{Definition}[section]
\newtheorem{remark}[definition]{Remark}
\newtheorem{conjecture}[theorem]{Conjecture}
\newtheorem{proposition}[theorem]{Proposition}
\newenvironment{proof}{{\bf Proof:}}{$\qed$\par}
\newenvironment{proofof}[1]{{\bf Proof of #1:}}{$\qed$\par}
\newenvironment{proofsketch}{{\sc{Proof Outline:}}}{$\qed$\par}

\usepackage{hyperref}
\hypersetup{
	bookmarksnumbered
}


	
	\begin{document}
		%{	\color{blue}   \textbf{Edit the parts in blue and remove this part}}
		\begin{center}
			\framebox{\parbox{6.5in}{
					{\bf{ELL888 Indian Institute of Technology Delhi} }\\ 
					{\bf  {\color{blue} {MCM Kernel Optimization}}}
					\\
{Scribed by: {\color{blue}\textit{Mridul Gupta (2021AIZ8322)}}\\ Instructors:
						 Sandeep Kumar and Jayadeva}
			}}
			\ \\
		\end{center}
%		\noindent{\bf Note}: {LaTeX template courtesy of UC Berkeley EECS dept.}
		
		\noindent {\bf Disclaimer}: {These notes have not been subjected to the
			usual scrutiny reserved for formal publications.  They may be distributed
			outside this class only with the permission of the Course Coordinator.}
		\vspace*{4mm}
		\setcounter{section}{\lecturenumber}
		%FOR SCRIBES: ---------- Begin Scribing Here ------------------------
	
\tableofcontents
\section{Structure}
Structure these points later
    \begin{itemize}
        \item[$\boxtimes$] Class separability might be worse in the feature space.
        \item Start with we want to get a bound on the risk on the test data. This
            requires knowing the distribution from which the data is being generated.
        \item We cannot know the distribution parameters. So, we estimate the actual risk
            using empirical risk which is the risk given some (training) samples from the
            distribution.
        \item Risk bound equation in VC slt.pdf
        \item Highlight two parts, the empirical risk and the structural risk
        \item make note that to reduce risk, we must reduce h (VC-Dimension)
        \item Mention VC-Dimension bound for $\Delta$-margin classifiers.
        \item Mention that in general it will be $n+1$ but it can be bound by reducing
            $\displaystyle \dfrac{R^2}{\Delta^2}$
        \item Show the simplified graph
        \item This transformation might not take to same dimensional space, might be
            higher dimensional. This function is a Kernel, that needs to be optimized to
            optimize structural risk.
        \item Also mention examples why sometimes even an infinite dimensional kernel
            might not work. And why we need Kernel Optimization.
        \item Go into the details of how to do this.
        \item Mention appropriate theorems, definitions and lemmas in between.
        \item Add a key takeaways section.
    \end{itemize}

Refined list to structure
    \begin{itemize}
        \item A little intro to MCM (slt, minimize vc dim bound, structural risk minimization).
        \item What is kernel optimization, why it is needed
        \item Amari's idea, magnifying around support vectors
        \item Xiong's idea of using a scatter matrix, Fisher discriminant
        \item Xiong's idea of creating a kernel scatter matrix, how it's equal to normal scatter
        \item Empirical feature space, matrix equations
        \item solving generalized eigenvalue problems
        \item Getting the MCM with optimized kernel
    \end{itemize}


\section{Introduction}
Support Vector Machines (SVMs) are older than the state of Haryana (the linear variant at
least), and the kernel version of SVMs were proposed by Vapnik \etal in 1992. Vapnik and
Chernvonenkis along with others also developed theory key thoeretical concepts that
constitute statistical learning theory. In what follows, we will discuss the theoretical
and practical advancements since then that led to the work titled ``Kernel optimization
using conformal maps for the minimal complexity machine'' by Jayadeva \etal.\par
Since this is the congruence of two paths, one leading to MCMs and the other leading to
kernel optimization in a data dependent way, in the following we first discuss MCMs, and
then kernel optimization, and finally how the two fit together.


\section{Where does MCM come from?}
The key concepts that lead to MCMs were available in statistical learning theory long ago,
but weren't applied until 2015. So let's discuss the key concepts that naturally lead to
minimal complexity machines.
\cite{mr:random}


\section{3}

	
	\bibliographystyle{plain}
	\begin{thebibliography}{10}
		\bibitem{mr:random}
	Cormen TH, Leiserson CE, Rivest RL, Stein C. 
		\newblock {\em Introduction to algorithms.}.
		\newblock MIT press.	2009 Jul 31
	\end{thebibliography}
	
	
	
\end{document}
